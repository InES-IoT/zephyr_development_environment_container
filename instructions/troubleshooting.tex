\begin{itemize}
  \item {\bf Running the Container fails after importing}

    You might need to use \mono{localhost} in the run command:
  \begin{monobox}
podman run --rm -it \
-v /path/to/project:/root/dev \
localhost/@\imagename{}@
\end{monobox}
  \item {\bf The WSL distro is not found after importing}

    Try rebooting your system.
  \item {\bf The WSL prints an unspecified error}

    Try to update the WSL with either of the following two commands:
  \begin{monobox}
wsl --update
wsl --update --web-download
\end{monobox}
  \item \textbf{WSL was imported as admin}

    When the WSL import is run as admin, the image is not available for other users, or the root file system of the WSL is not available for other users.
    The solution in this case is to close the current WSL terminal and open in from a new terminal.
  \item {\bf The debug probe keeps announcing itself as mass storage device}

    Open the J-Link shell an disable the mass storage feature.
  \begin{monobox}
JLink
msddisable
\end{monobox}
\end{itemize}
