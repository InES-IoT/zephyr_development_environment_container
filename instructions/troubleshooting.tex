\begin{itemize}
  \item {\bf Running the Container fails after importing}

    You might need to use \mono{localhost} in the run command:
  \begin{monobox}
podman run --rm -it \
  -v /path/to/project:/root/dev \
  localhost/@\imagename{}@
\end{monobox}
  \item {\bf The WSL distro is not found after importing}

    Try rebooting your system.
  \item {\bf The WSL prints an unspecified error}

    Try to update the WSL with either of the following two commands:
  \begin{monobox}
wsl --update
wsl --update --web-download
\end{monobox}
  \item {\bf The WSL prints \mono{CreateProcessParseCommon: Failed to translate} errors.}

    Try the following fixes.
    \begin{itemize}
      \item Try removing and then importing the image again.
        The following command removes the image.
        \begin{monobox}
wsl --unregister @\imagename{}@
\end{monobox}
      \item The file system might not be available from within WSL.
        Run the following command to list the available file systems.
        \begin{monobox}
@\cmdinwsl{}@ ls /mnt
\end{monobox}
        You can try mounting the missing file system (\mono{D:\\} in this example) with the following commands.
        \begin{monobox}
@\cmdinwsl{}@ mkdir /mnt/d
@\cmdinwsl{}@ mount -t drvfs D: /mnt/d
\end{monobox}
      \item Check for suspicous entries in the \mono{Path} environment variable.
        \begin{monobox}
rundll32 sysdm.cpl,EditEnvironmentVariables
\end{monobox}
    \end{itemize}
  \item {\bf The debug probe keeps announcing itself as mass storage device}

    Open the J-Link shell an disable the mass storage feature.
  \begin{monobox}
JLink
msddisable
\end{monobox}
\end{itemize}
