\section{Setup with a Container Engine}
\label{setupwithcontainerengine}

Follow these steps to set up a Zephyr development environment with a container runtime.

\begin{itemize}
  \item Install the \emph{J-Link Software and Documentation pack (\href{https://www.segger.com/downloads/jlink}{Download})}.
  \item Install a serial terminal like \href{https://formulae.brew.sh/formula/picocom}{picocom}.
  \item Install a container engine (we recommend \href{https://podman.io/docs/installation}{podman}).
    \begin{infobox}
      On MacOS 26 and later we recommend using \href{https://opensource.apple.com/projects/container/}{Apple's \mono{container} tool}.
      Install it using the \mono{container-X.X.X-installer-signed.pkg} file from \href{https://github.com/apple/container/releases}{here}.
      Just substitute \mono{podman} with \mono{container} in the following commands.
    \end{infobox}
  \item Load the image.
        \begin{monobox}
podman image load --input @\imagename{}@_ARCH.tar.gz
\end{monobox}
  \item Execute the following command to run and enter the container.
        \begin{monobox}
podman run --rm -it \
  -v /path/to/project:/root/dev \
  localhost/@\imagename{}@
\end{monobox}
    With the \mono{-v /path/to/project:/root/dev} argument, the \mono{/path/to/project} path on the host is made available inside the conainer at \mono{\~/dev}.
\end{itemize}
