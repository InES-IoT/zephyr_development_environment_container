\section{Using the Zephyr Environment}
\label{usingenv}

\subsection{Build Sample}

Run the following inside the environment to compile the blinky sample.

\begin{itemize}
  \item For \textbf{WSL}:
  \begin{monobox}
cd ~/zephyrproject/zephyr/samples/basic/blinky/
west build --pristine --board @\board{}@ --build-dir /tmp/build
\end{monobox}
  \item For \textbf{Podman/Docker}:
  \begin{monobox}
cd ~/zephyrproject/zephyr/samples/basic/blinky/
west build --pristine --board @\board{}@ --build-dir ~/dev/build
\end{monobox}
\end{itemize}

\begin{infobox}
  When using WSL, we recommend to include the \mono{--build-dir /tmp/build} argument.
  If omitted the build system will compile the application in a \mono{build} folder in the current directory, which will be very slow when this directory is on the Windows file system (e.g. in \mono{/mnt/c/}).
\end{infobox}

\begin{infobox}
  When using Podman/Docker, setting the build directory to \mono{\~/dev/build} is necessary, otherwise you cannot access the binary from outside the container.
\end{infobox}

% container engine: make sure builddir accessible

\newpage

\subsection{Flash Sample}

Start \emph{J-Flash Lite}, select the target device (\texttt{\mcu}), and make sure the \emph{Internal flash} flash bank is selected.

\begin{center}
  \begin{tikzpicture}
    \definecolor{gray}{rgb}{.941,.941,.941}
    \node [anchor=south west] {\includegraphics[height=8cm]{jflashlite1.png}};
    \draw (6.99,6.43) [red,ultra thick] circle[radius=.4];
    \draw (.79,3.08) [red,ultra thick] circle[radius=.4];
    \draw (.66,6.55) [gray,fill] rectangle (5,6.3);
    \node at (.52,6.42) [anchor=west] {\fontsize{8pt}{8pt}\selectfont\mcu};
  \end{tikzpicture}
\end{center}

In the next window select the firmware file (\mono{build/zephyr/zephyr.hex}) and click on \emph{Program Device}.

\begin{infobox}
  The WSL file system can be accessed in the explorer.
  \begin{center}
    \includegraphics[width=.5\paperwidth]{explorer_wsl_hex}
  \end{center}

  It is possible, that the \emph{Linux} file system does not show up.
  If this is the case, it can be accessed by manually entering its path in the explorer:
  \mono{\\\\wsl\$}
  \begin{center}
    \includegraphics[width=.65\paperwidth]{win_file_explorer_wsl.png}
  \end{center}
  Press \emph{enter} to access the WSL file system.
\end{infobox}
