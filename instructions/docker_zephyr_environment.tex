\documentclass[11pt,a4paper,notitlepage]{article}

\usepackage[margin=3cm]{geometry}

\usepackage{bookmark}

\usepackage{hyperref}
\hypersetup{colorlinks=true,urlcolor=blue,linkcolor=black}

\usepackage{enumitem}
\setlist[enumerate]{leftmargin=*}
\setlist[itemize]{leftmargin=*}

\usepackage{listings}
\lstset{
  basicstyle=\ttfamily,
  columns=flexible,
  escapechar=@,
  tabsize=4,
  commentstyle=\color{darkgray},
  frame=l,
  framerule=1pt
}
\NewDocumentCommand{\mono}{m}{\lstinline{#1}}
\lstnewenvironment{code} {\lstset{language=C}} {}

\usepackage{calc}

\usepackage{float}

\usepackage{booktabs}

\usepackage{xcolor}

\usepackage{tikz}

\setlength{\parindent}{0pt}
\setlength{\parskip}{.5em}
\setlength{\skip\footins}{2em}

\renewcommand*\familydefault{\sfdefault}

\usepackage{graphicx}
\graphicspath{{img/}}

\usepackage{tcolorbox}
\tcbset{parbox=false}
\newenvironment{infobox} {
\begin{tcolorbox}[title=Info] } {
\end{tcolorbox} }

% \def\zephyrversion {3.6.0}
% \def\sdkversion {0.16.5}
% \def\mcu {STM32F429ZI}
% \def\board {stm32f429i\_disc1}
% \def\imagename {zephyr\_v\zephyrversion}

\ifundef{\zephyrversion}{\errmessage{error: zephyrversion variable not set}}
\ifundef{\sdkversion}{\errmessage{error: sdkversion variable not set}}
\ifundef{\mcu}{\errmessage{error: mcu variable not set}}
\ifundef{\board}{\errmessage{error: board variable not set}}
\ifundef{\imagename}{\errmessage{error: imagename variable not set}}

\def\cmdinwsl {wsl -d \imagename{}}

\def\maintitle {Zephyr Development Environment Container}

\NewDocumentCommand{\puttitle}{m}{%
  \begin{center}
    \huge
    \begin{minipage}[b]{.4\textwidth}
      \includegraphics[height=2.5em]{en-zhaw-ines}
    \end{minipage}%
    \begin{minipage}[b]{.6\textwidth}
      \begin{flushright}
        \textbf{\maintitle}\\
        #1
      \end{flushright}
    \end{minipage}%
    \vspace{1cm}
  \end{center}
  \lfoot{\maintitle{} #1}
}

\usepackage{fancyhdr}
\pagestyle{fancy}
\lhead{}
\chead{}
\rhead{}
\lfoot{}
\lfoot{}
\cfoot{}
\rfoot{\thepage}
\renewcommand{\headrulewidth}{0pt}
\renewcommand{\footrulewidth}{0pt}


\begin{document}

\puttitle{}

\section{Introduction}

\newpage

\section{Windows}

Install the following programs.

\begin{itemize}
  \item \textbf{Windows Terminal (\href{https://aka.ms/terminal}{Download})}
  \item \textbf{J-Link Software and Documentation pack
          (\href{https://www.segger.com/downloads/jlink/JLink_Windows_x86_64.exe}{Download})}

        Tick the \emph{Install USB Driver for J-Link (requires admin rights)} checkbox
        during installation.
  \item \textbf{podman for Windows (podman-X.X.X-setup.exe)
          (\href{https://github.com/containers/podman/releases}{Download})}

        When installing podman, make sure that the \emph{Install WSL if not present} checkbox is checked.

        The following instructions are based on the
        \href{https://github.com/containers/podman/blob/main/docs/tutorials/podman-for-windows.md}
        {official
          instructions}.

        Run the following commands after the podman installation is complete.

        \begin{lstlisting}
podman machine init
podman machine start
podman run ubi8-micro date
       \end{lstlisting}

        \begin{infobox}
          If the \emph{podman run} command produces a gigantic error message try
          reinstalling the WSL image by running the following command and then
          repeating the commands from above.

          \mono{podman machine rm}
        \end{infobox}
        \begin{infobox}
          The \mono{podman machine start} command needs to be re-run after a reboot.
        \end{infobox}
\end{itemize}

\newpage

\section{Using the Container}

\subsection{Load the Image}

\mono{podman load --input} {\ttfamily\imagename{}}\mono{.tar.gz}

\subsection{Start Shell in Container}

\begin{lstlisting}
podman run --rm -it --network="host" `
  -v D:\path\to\project:/root/dev `
  @\ttfamily\imagename{}@
\end{lstlisting}

\begin{infobox}
  In case you get a
  \emph{D:\textbackslash{}path\textbackslash{}to\textbackslash{}project: no such
    file or directory} error the file system might not be available from the WSL.
  Run the following command to list the available file systems.

  \mono{wsl -d podman-machine-default ls /mnt}

  You can try mounting the missing file system (\mono{D:\\} in this case) with
  the following commands.

  \mono{wsl -d podman-machine-default sudo mkdir /mnt/d}

  \mono{wsl -d podman-machine-default sudo mount -t drvfs D: /mnt/d}
\end{infobox}

\subsection{Build Sample}

\mono{cd ~/zephyrproject/zephyr/samples/basic/blinky/}

\mono{west build --pristine --board stm32f429i_disc1}

\begin{infobox}
  Append the \mono{--build-dir /tmp/build} argument if you mounted a drive from
  the Windows host. This will speed up the build process as accessing the
  Windows file system from the WSL is slow.

  \mono{west build --pristine --board stm32f429i_disc1 --build-dir /tmp/build}

  This argument will then also be needed when running \mono{west flash}.
\end{infobox}

\subsection{Flash Sample}

To flash from within the WSL the J-Link Remote Server is used. Run the following
command in a PowerShell to get the IP of the WSL network adapter.

\mono{wsl -d podman-machine-default grep nameserver /etc/resolv.conf}

Start the \emph{J-Link Remote Server}

\begin{infobox}
  If the J-Link Remote Server shows an error try using a different port.
\end{infobox}

run the following command to flash the board.

\mono{west -v flash --runner jlink --tool-opt='-ip 172.21.160.1:19020'}

\begin{infobox}
  If you get a \emph{FAILED: Can not connect to J-Link via TCP/IP} error try
  adding a firewall rule that allows inbound connections on the WSL network
  adapter. To do that run the following command in an admin PowerShell.

  \begin{lstlisting}
New-NetFirewallRule -DisplayName "WSL" -Direction Inbound `
 -InterfaceAlias "vEthernet (WSL)" -Action Allow
  \end{lstlisting}

  Firewall rules can also be managed by running \mono{wf.msc} in a PowerShell.
\end{infobox}

\newpage

\newgeometry{margin=1cm,bottom=2cm}
\fancyfoot{}

\appendix

\section{Containerfile}\label{containerfile}
\lstinputlisting[basicstyle=\ttfamily\footnotesize]{../Containerfile}

\end{document}

% Local Variables:
% TeX-output-dir: "build"
% TeX-master: t
% End:
